\documentclass[a4paper, 10pt]{article}

\usepackage{geometry}
\usepackage{hyperref}

\geometry{hmargin=2.5cm, lmargin=3cm, rmargin=2cm}

\title{Notes and Logbook}
\author{Sinan DAROUKH}
\date{4th May 2019}

\begin{document}

\begin{titlepage}
\maketitle
\end{titlepage}

\tableofcontents
\newpage

\section{Objectifs et mise en contexte du stage}
\subsection{The CERN and the LHCb Departement}
\subsection{What is WinCC-OA ?}
\subsection{What is JCOP ?}
\subsection{Main goals of the internship}
\section{Avancements des travaux}
\section{Logbook and tasks done}
\footnotesize
\subsection{Iteration from the Monday 4th of May to the Friday 15th of May}
\paragraph{Objectives of the iteration :} 
The main goal of this iteration was to get more familiar with WinCC-OA and the JCOP Framework. Also the main idea was to think about how to make the LHCb experiments monitor more accessible from the web without changing the WinCC-OA pre-existing project drastically.
\paragraph{Tasks done \& notes :}
\begin{itemize}
    \item Reading documentations about WinCC-OA.
    \item Setting up the main project.
    \item Reading/Writting emails to the ISIMA supervisor (Mr HILL) and the CERN one (Mr Luis Granado Cardoso).
    \item Helping and exchanging with Loann, who is working on the same technology as me.
    \item Going through the tutorial slides.
    \item Experimenting through Exercice 1,2,3 and 4.
    \item Facing some difficulties with the CTRL system.
    \item Reading documentation about HttpServer
    \item Watching videos about WinCC-OA on \href{https://www.youtube.com/user/ETM2011}{Siemens} and  \href{https://www.youtube.com/channel/UCGBnHd1-B-Zg9MDsjTk0-Sw}{KAASM}'s youtube channel. 
    \item Doing daily meetings with Luis to track the progess.
\end{itemize}

\subsection{Iteration from the Wednesday 15th of May to the Sunday 20th of May}
\paragraph{Objectives of the iteration :}
The aim of this iteration was to look at all the documentation about the WebServer components in WinCC-OA, how to navigate properly between panels, how to manage users access permissions and restrictions, and how to well organise configs files in a project.
\paragraph{Tasks done \& notes :}
\begin{itemize}
    \item I was sick during a part of this iteration, so I did not work effictively on Monday
    \item Reading all the documentation
    \item 
\end{itemize}

Configs files notes : 
config.level is for managers configurations, and basically, loading libs through this file.
config.redu is for redundancy, that we are not seeking for, at the moment
config.http is a premade file from the WinCC-OA folder
config.webclient is an additional file for configuration Desktop and Mobile UI add-ons\dots Not what we are looking for

I have tested multiple stuffs on those. Nothing really can simply the actual config (root) file.
I guess there's another option not yet tested which is the own conf file, idk if we can use more than once.

For the navigation fluidify and simply, I have been testing stuffs and reading about the differences between Modules, Embedded Modules , Child and Root Panel.
To navigate properly, I think I should take a look at the Topologies panels components...
Embedded modules through one root module, may be the best option.

\subsection{Iteration from Wednesday 20th to the 27th}
TO DO : 
\begin{itemize}
    \item User permissions, one can connect to view, another can connect to administrate
    \item Play with the alarm Screen, make a shortcut to it
    \item (FSM) Implement one, how it goes
\end{itemize}
DONE :  
\begin{itemize}
    \item Login panel has been implemented (worked on both ULC and Regular WinCC)
    \item Can access on user admnistration panel through it : Module > SysMgm > Permissions > User Admnistration, which on root access list all users and permissions
    \item Groups > Admnistrate > Permissions : You can change the group permissions but also, create new groups with new permissions
    \item Permissions are logged in an Authorization Bits system. The first five bits are already define, they are predefined and un-changeable, but you can change the description if you want and texts of it : 
    \begin{itemize}
        \item 1 : Visualisation: Visualize only
        \item 2 : Normal operator authorization: permits the opening of child panels.
        \item 3 : Advanced operator authorization: permits execution of commands, explicit setting of replacement values, input of correction values as well as changes to all value range types.
        \item 4 : Administration: permits the use of the PARA.
        \item 5 : Acknowledgement: permits acknowledgment of alerts.
        \item 32 : Allows SSO for one work station
    \end{itemize}
    \item Can also access on login statistics panel to see whose connected : Module \textgreater\  Login Statistics
    \item We can manage the Components access thanks to the boolean getUserPermission() function : CONTROL \textgreater\  Control functions \textgreater\ G \textgreater\ getUserPermission()
    \item We can check on the Main Panel, by Login via Guest or via Root
    \item Auto login done, with inactivity (Glitch with inactivity, security one) (UI number changed some time)
    \item Alarm Panel, I have worked on it, but there is a glitchy features, the windows appears behind the Alarm Screen
    \item I also haven't the time to experiment on the FSM Things
\end{itemize}

\subsection{Iteration from Wednesday 27th to Tuesday 2nd}
The aim of the iteration was to :
\begin{itemize}
    \item Adding and testing the FSM, on the Web app, it actually works but...
    \item Create Shortcut for opening differents kinds of panels, I had issues with the DEN Panel... But now it works
    \item Didn't check on the Alarm Screen and the Users Permissions, as I had issues with the DNS part of the tuturials.
\end{itemize}


\section{Real Gantt chart}

\end{document}