\documentclass[../main.tex]{subfiles}

% TO DO 
% READ AGAIN THE TEXT
% ADD PICTURES FOR LHC / LHCb
% ADD GANTT DIAGRAMM / COMMENT THEM

\begin{document}

\chapter{The CERN and its experiments}
\section{The CERN}
\paragraph{}
The Frenchman Louis de Broglie, who was awarded the Nobel Prize in physics in 1929, was behind the creation of CERN. 
In 1949, he proposed the creation of a European scientific laboratory in order to breathe new life into scientific research following the Second World War.
\par \noindent \newline
In 1952, with the support of UNESCO, the European Council, with the support of the United Nations Educational, Scientific and Cultural Organization (UNESCO), decided to set up a European scientific laboratory for Nuclear Research (CERN), it is the result of an agreement between eleven European governments. The municipality of Meyrin, located near Geneva, was chosen to host this laboratory.
\par \noindent \newline
In 1954, the first construction work on the site began and on the 29 of September the CERN Convention was signed by twelve European States. It was at this date when the research centre, then called the European Organisation for Nuclear Research, was officially established.
Today CERN is very well known thanks to its experiences and more particularly its particle accelerator.
\par \noindent \newline
However, it should be noted that many accelerators have succeeded one another on the laboratory site. 
The first was the Synchro-Cyclotron a protons inaugurated in 1957. The accelerators being larger and larger, an agreement was made with France in 1965 to expand the site on French territory. 
Then in 1981 it was decided to build an accelerator in a tunnel with a circumference of 27 kilometers located 100 meters deep underground. 
This tunnel, which is housed in first the LEP (Large Electron Positron Collider). This one was replaced by the current LHC (Large Hadron Collider) in 2008.
\par \noindent \newline
CERN now has 23 member countries and around 2,500 employees and more of 17,500 researchers who come to the Meyrin site to carry out experiments on the premises of the research centre. 
The budget of such an organisation amounts to more than 1 billion Swiss Francs per year.
This funding is fully covered by the Member States.

\subsection{The LHC - Large Hadron Collider}
\paragraph{}
The Large Hadron Collider (LHC) is the world’s largest and most powerful particle accelerator. It first started up on 10 September 2008, and remains the latest addition to CERN’s accelerator complex. The LHC consists of a 27-kilometre ring of superconducting magnets with a number of accelerating structures to boost the energy of the particles along the way.
\par \noindent \newline
Inside the accelerator, two high-energy particle beams travel at close to the speed of light before they are made to collide. The beams travel in opposite directions in separate beam pipes – two tubes kept at ultrahigh vacuum. They are guided around the accelerator ring by a strong magnetic field maintained by superconducting electromagnets. The electromagnets are built from coils of special electric cable that operates in a superconducting state, efficiently conducting electricity without resistance or loss of energy. This requires chilling the magnets to ‑271.3°C – a temperature colder than outer space. For this reason, much of the accelerator is connected to a distribution system of liquid helium, which cools the magnets, as well as to other supply services.
\par \noindent \newline
Thousands of magnets of different varieties and sizes are used to direct the beams around the accelerator. These include 1232 dipole magnets 15 metres in length which bend the beams, and 392 quadrupole magnets, each 5–7 metres long, which focus the beams. Just prior to collision, another type of magnet is used to "squeeze" the particles closer together to increase the chances of collisions. The particles are so tiny that the task of making them collide is akin to firing two needles 10 kilometres apart with such precision that they meet halfway. All the controls for the accelerator, its services and technical infrastructure are housed under one roof at the CERN Control Centre. From here, the beams inside the LHC are made to collide at four locations around the accelerator ring, corresponding to the positions of four particle detectors – ATLAS, CMS, ALICE and LHCb.

\subsection{The LHCb - Large Hadron Collider beauty}
\paragraph{}
The Large Hadron Collider beauty (LHCb) experiment specializes in investigating the slight differences between matter and antimatter by studying a type of particle called the "beauty quark", or "b quark".
Instead of surrounding the entire collision point with an enclosed detector as do ATLAS and CMS, the LHCb experiment uses a series of subdetectors to detect mainly forward particles - those thrown forwards by the collision in one direction. The first subdetector is mounted close to the collision point, with the others following one behind the other over a length of 20 metres.
\par \noindent \newline
An abundance of different types of quark are created by the LHC before they decay quickly into other forms. To catch the b quarks, LHCb has developed sophisticated movable tracking detectors close to the path of the beams circling in the LHC. The 5600-tonne LHCb detector is made up of a forward spectrometer and planar detectors. It is 21 metres long, 10 metres high and 13 metres wide, and sits 100 metres below ground near the village of Ferney-Voltaire, France. About 700 scientists from 66 different institutes and universities make up the LHCb collaboration (October 2013).

\chapter{Internship context}
\section{Internship subject presentation}
\paragraph{}
As you have previously read, the LHCb uses a series of subdetectors, which means that LhHCb's experiments controls systmes handles the configuration, monitoring, and operation of all experimental equipment involved in the various activities of the experiment.
\par \noindent \newline
Millions of parameters originating from a large variety of equipment, ranging from commercial power supplies to sophisticated home made electronics, have to be collected, stored and presented to the physicists operating the experiment. The scale of the system requires the control system to run distributed over hundreds of computers in a coherent and coordinated, hierarchical, fashion.
\par \noindent \newline
A commercial industrial-strength SCADA (Supervisory Control and Data Acquisition) System - Siemens WinCC-OA - has been chosen as the basis for the development. WinCC-OA has been complemented by another tool - SMI++ - combining a rule-based approach with Finite State Machine methodology, providing a very convenient mechanism for the modelling and
automation of large scale, high complexity, installations. 
\par \noindent \newline
As a CERN intern, my mission was to investigate, tests and deploy a mechanism to make available the WinCC-OA User Interfaces on the web, in order to provide easy access to the Controls Systems informations to the community of the LHCb users. To streamline the access to the Control Systems informations was my main internship goal.

\section{Working methods}
\subsection{COVID-19 and teleworking}
\paragraph{}
Coronaviruses (CoV) are a large family of viruses. COVID-19 is a new strain of coronavirus that causes illness ranging from the common cold to more severe diseases. It was firstly discovered in December 2019 at Wuhan in China, but quickly spread around the world, and thereby evolving to the status of a pandemic virus. Due to this pandemic, multiples countries were forced to close their borders, and moving abroad was not recommended or strictly prohibited.
\par \noindent \newline
On 12 March, the French President Macron announced on public television that all schools and all universities would close from Monday 16 March until further notice. On 16 March, Macron announced mandatory home confinement for 15 days starting at noon on 17 March. This was extended twice, and ended on 11 May. Furthermore, a few days before, the UCA, Clermont-Ferrand university, announced that all internships were now prohibited, until this pandemic situation calm down.
\par \noindent \newline
A lot of ISIMA's internships were cancelled or postponed, but the CERN agreed to maintain mine under one condition : the whole internship should be done in remote to prevent the virus propagation.
Thus, the duration of my internship, which was initially 5 months long, was shorten to 3 months. Indeed, some times was needed in order to set up the teleworking infrastructure, but I was also no longer able to work in August.

\subsection{Workflow}
\paragraph{}
As my internship has to be done remotely, we opted for a tele-working friendly workflow and used something that can look like a AGILE methods but which not really one.
We proceed throught iterations, at each one of them, my CERN supervisor, Mr Luis Granado Cardoso gave me some tasks to do or some topics to investigates. Then we meet at the end of the iteration and discuss about what has been done during the iteration, what need to be done, and what need to be improve or more investigate for the next one.
\par \noindent \newline
Here is a gantt diagram about the various iterations we had during the internship.
% TO DO PUT DIAGRAM
To know more about the contents of each iterations, refers to the appendices where you would fine my internship logbook.
\subsection{Tools and languages used}
\paragraph{}
To work properly and efficiently, we decided to use various tools to communicate and stay in touch even though we were working remotely. Most of them were provided by the CERN.
\paragraph{Zoom}
Zoom is an american communications technology company. It provides videotelephony and online chat services through a cloud-based peer-to-peer software platform and is used for teleconferencing, telecommuting, distance education, and social relations. At the CERN, we use the paid version to be allowed to do call longer than 40 minutes. Zoom was our online conference meeting provider for our weekly meeting.
\paragraph{Mattermost}
Mattermost is an open-source, self-hostable online chat service with file sharing, search, and integrations. It is designed as an internal chat for organisations and companies, and mostly markets itself as an open-source alternative to Slack and Microsoft Team. The CERN host their own Mattermost servers, and my supervisor and I used this chat service to exchange between our online meeting.
\paragraph{Outlook}
Outlook is a personal email provider and email manager, CERN use it as a manager. With the email address, I have been given, I was able to stay tuned about the LHC activities, but also exchange with others LHC workers. 
\paragraph{\LaTeX}
\LaTeX\ is a document preperation system, using a markup tagging conventions to define the general structure of a document. As it is widely used in academia, for the publication of scientific or report, I have decided to write my report, notes and logbook with it.
\paragraph{WinCC-OA}
SIMATIC WinCC Open Architecture is a SCADA system for visualizing and operating of processes, production flows, machines and plants in all lines of business. It has been chosen by the CERN at the basis of the development. 
\par \noindent \newline 
The next part of the report will cover in more detail what is WinCC-OA and how it works.

\end{document}