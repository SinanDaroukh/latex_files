\documentclass[a4paper, 10pt]{article}

\usepackage{geometry}

\geometry{hmargin=2.5cm, lmargin=3cm, rmargin=2cm}

\title{Notes and Logbook}
\author{Sinan DAROUKH}
\date{4th May 2019}


\begin{document}

\begin{titlepage}
\maketitle
\end{titlepage}

\tableofcontents
\newpage

\section{Internship Main Goal}
\section{Avancements des travaux}
\section{Etapes réalisées - Logbook}
\footnotesize
\subsection*{Iteration from the 4th May to the 15th May - Two first weeks}
\begin{itemize}
    \item Reading documentations about WinCC-OA.
    \item Setting up the main project.
    \item Reading/Writting emails to the ISIMA supervisor (Mr HILL) and the CERN one (Mr Luis Granado Cardoso).
    \item Helping and exchanging with Loann, who is working on the same technology as me.
    \item Going through the tutorial slides.
    \item Experimenting through Exercice 1,2,3 and 4.
    \item Facing some difficulties with the CTRL system.
    \item Reading documentation about HttpServer
    \item Watching videos about WinCC-OA on KAASM's youtube channel.
    \item 
\end{itemize}



\subsection*{Friday 15th May}
\begin{itemize}
    \item Starting work at 09.30
    \item Meeting with Luis searching about other stuffs WebServer/HttpServer
    \item Things to do : Look at all the documentation about WebServer, Navigations, Permissions and access (restrictions) and configs files (configs levels).
    \item Ending work at 14.30
\end{itemize}

\subsection*{Monday 18th May}
\begin{itemize}
    \item Starting work at 15.00
    \item Sick as fuck
    \item Ending work at 17.00
\end{itemize}

\subsection*{Tuesday 19th May}
\begin{itemize}
    \item Starting work at 10.00
    \item Finished the documentation.
    \item 
    \item Ending work at 17.00
\end{itemize}

\subsection{Iteration from Wednesday 15th to the 20th}

Configs files notes : 
config.level is for managers configurations, and basically, loading libs through this file.
config.redu is for redundancy, that we are not seeking for, at the moment
config.http is a premade file from the WinCC-OA folder
config.webclient is an additional file for configuration Desktop and Mobile UI add-ons\dots Not what we are looking for

I have tested multiple stuffs on those. Nothing really can simply the actual config (root) file.
I guess there's another option not yet tested which is the own conf file, idk if we can use more than once.

For the navigation fluidify and simply, I have been testing stuffs and reading about the differences between Modules, Embedded Modules , Child and Root Panel.
To navigate properly, I think I should take a look at the Topologies panels components...
Embedded modules through one root module, may be the best option.

\subsection{Iteration from Wednesday 20th to the 27th}
TO DO : - User permissions, one can connect to view, another can connect to administrate < DONE >
        - Play with the alarm Screen, make a shortcut to it
        - (FSM) Implement one, how it goes

DONE :  - Login panel has been implemented (worked on both ULC and Regular WinCC)
        - Can access on user admnistration panel through it : Module > SysMgm > Permissions > User Admnistration, which on root access list all users and permissions
        - Groups > Admnistrate > Permissions : You can change the group permissions but also, create new groups with new permissions
        - Permissions are logged in an Authorization Bits system. The first five bits are already define, they are predefined and un-changeable, but you can change the description if you want and texts of it : 
            - 1 : Visualisation: Visualize only
            - 2 : Normal operator authorization: permits the opening of child panels.
            - 3 : Advanced operator authorization: permits execution of commands, explicit setting of replacement values, input of correction values as well as changes to all value range types.
            - 4 : Administration: permits the use of the PARA.
            - 5 : Acknowledgement: permits acknowledgment of alerts.
            - 32 : Allows SSO for one work station
        - Can also access on login statistics panel to see whose connected : Module > Login Statistics
        - We can manage the Components access thanks to the boolean getUserPermission() function : CONTROL > Control functions > G > getUserPermission()
        - We can check on the Main Panel, by Login via Guest or via Root
        - Auto login done, with inactivity (Glitch with inactivity, security one) (UI number changed some time)

        - Alarm Panel -> 

        - FSM haven't the time

\subsection{Iteration from Wednesday 27th to Tuesday 2nd}
The aim of the iteration was to :
    - Adding and testing the FSM, on the Web app, it actually works but...
    - Create Shortcut for opening differents kinds of panels, I had issues with the DEN Panel...
    - Didn't check on the Alarm Screen and the Users Permissions, as I had issues with the DNS part of the tuturials.
\section{Diagrammes de Gantt Réel}

\end{document}