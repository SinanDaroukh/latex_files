\documentclass[a4paper, 10pt]{article}

\usepackage{makeidx}
\usepackage{formular}
\usepackage{listings}
\usepackage[utf8]{inputenc}
\usepackage[dvipsnames]{xcolor}

\usepackage{mdframed}
\usepackage{multicol}
\usepackage{hyperref}
\usepackage{geometry}
\usepackage{mdframed}
\usepackage{listings}

\geometry{hmargin=2.5cm, lmargin=3cm, rmargin=2cm}

\title{Notes and Logbook}
\author{Sinan DAROUKH}
\date{4th May 2019}

\begin{document}

\begin{titlepage}
\maketitle
\end{titlepage}

\tableofcontents
\newpage

\section{WinCC OA \&  The JCOP Framework ?}
\subsection{What is WINCC-OA ?}
WinCC-OA is the Supervisory Control And Data Acquisition (aka SCADA) system chosen by JCOP.
WinCC-OA aims is mainly to control and acquire data from the sensors from reals controls on experiments.
PVSS was a SCADA system, made by ETM. Siemens now owns ETM and rebranded PVSS as WinCC-OA, but it's still the same tool.
It's a tool for building SCADA applications

\subsection{What is JCOP ?}
JCOP stands for Joint COntrols Project which is a collaboration between the LHC experiments, the PH Departement and EN-ICE, the Controls Group in the Engineering Departement. JCOP aims to reduce the overall manpower cost required to produce and run the experiment control systems.

\subsection{What is JCOP Framework ?}
The JCOP Framework provides all the components required for WinCC-OA tool. Basically, it's a layer of software components.

\subsection{What is FSM ?}
FSM stands for Final State Machine, it's abstract representation of your experiment.


\subsection{Exercice \#1}
I have been trying http access into PMON with devwinccweb02:4999 on the lbts machine, it works fine.
However I do not understand how the layout thing work so I better ask Luis.

\subsection{Exercice \#2}
I have done the exercice 2, slide 127 is useful to understand the architecture of DataPoint (DP)
Note to self : Make a scheme about DPT, DP and DPE
All the process to create a DataPoint Type, instanciate a DataPoint is in here.

\subsection{Exercice \#3}
Original value from the PARA editor :\_original..\_value
PARA is very useful to debug and initialize :\_original..\_value
Learn about callback function, how to initialize components, also how to generate script using the wizard.

\subsection{Exercice \#4}

Slide 165 : A note on dynamic arrays
dyn\_int, dyn\_errClass, dyn\_float are just dynamic arrays of int, errClass, floats, etc \dots 
Length of the dynamic array can be check using dynlen() function.
errClass is a class dealing with special WinCC err.

Slide 166 : anytype/mixed types
anytype type enables developers to write functions that accept parameters or arbitrary type (very simple generic programming). Once an anytype variable is written to, it starts to behave as if it was of a type of the literal/variable that was assigned to it.

Pro-tips : Use const references, to avoid needless copying of variables in callback functions

\subsection{Extending Exercise \#4}

Interesting point of view of why we should not try to save space doing interface but caring about color blind people.
Slide 187 : Basic DP access functions
dpGet() - get the value of a datapoint config attribute once.
dpConnect() - register a callback to be executed whenever a datapoint config attribut changes.
dpSet()/dpSetWait() - set the value of a datapoint config attribute.\\

Check also for dpQuery(), dpDelete(), etc\dots

Slide 188 : Performance
dpSetWait() - set around ~1000 datapoints in a single function call.
dpGet() - request values of as many datapoints as possible (pratical) in a single function call.

Executing them for a single datapoint, is one the biggest performance killers in WinCC OA ! Warning !

Slide 194 : More Aesthitics (Aesthitics standards to check)

Slide 197 : Do not add computations, e.g. comparing voltage levels in panel code instead put those functions in CONTROL libraries. UI should only do UI stuffs.

Slide 203 : Rewiew - Storage of values
There are 3 different “places” where we can storevalues.\\
You need to distinguish:
\begin{itemize}
    \item A number in a data point element.\\
DPs (and hence DPEs) are held in the Event Manager process’s memory space.
    \item A number in a widget on a panel.\\
Widgets are held in a User Interface Manager.
    \item A CONTROL variable in a script behind a widget.
\end{itemize}

Remember to name every widget, else you won't be able to access them from other widgets panel.
(Check Slide for conventionnal names 205)

DP Name Syntax in Script (Slide 208 - Good one, very important)

\section{WinCC OA \&  The JCOP Framework Course - Part 2}
\subsection{Exercice \#5}
Slide 10 : Concept of a reference panel which can be useful and time-saver.
Draw it once, but instanciate it many times.

To make a "normal" panel as a reference panel, we have edit the previous script we made and make them more generic with \$parameters.

What happens if you change the reference panel and then re-open the parent panel ?
Changes are applied to the parent panel.

How could you change it again to require that the user must supply a system name (dist\_nn:) each time he makes a new panel instance?
By using the same process as the \$parameters for the distro.

\subsection{Properties}
Performance penalty
Defined in the ScopeLib of each panels.
Cleaner way to pass parameters to the reference panel.

\section{Exercice \#5 - Scripts and CTRL Manager}
Scripts are contained in the Script folder, they run in the background, we need to associate them a CTRL manager to run.
If we edit a script, which already associated with a CTRL manager, we need to restart the CTRL manager.

Slides 57-61 ARE IMPORTANT

\section{Exercice \#6 - Debugging Tips}
\begin{itemize}
    \item Look the logviewer
    \item Check Manager Status
    \item Use PARA
    \item Use DebugTN() and DebugFTN()
\end{itemize}
DebugTN("") : DebugTN(“Entering with pressure=” + pressure);\\
DebugFTN(""): DebugFTN(“myDebugFlag","Custom debug myDebugFlag is activated.“);\\
DebugFTN("") need setting up debugging flag to the control manager. -dbg myDebugFlag
Usefuls variables for debugging :\\
\_\_LINE\_\_ : Give the LINE in the code where is generate the debug message\\
\_\_FUNCTION\_\_ : Give the name of the function which called the Debug function\\
\_\_FILE\_\_ : Give the file that contained the function call.

Got issues running the CTRL Script and associate it to the Manager.
Gonna try with own prgm and not solution tomorrow.

\subsection{WinCC OA Exceptions}
Error Class : errClass
Can use : try { }, catch { }, finally { } and throw(err)
Online help : Control > Introduction CONTROL > Error handling

\subsection{Alarms / Alerts}
Called in PVSS : Alerts. WinCC name is Alarms.
2 parts : Alert handling, how the alert should be raised and Alert Class, attributes which generally apply to more than one alert.

To add an alert, go to PARA, right on the DPT you'd like to add a alert, then click on "Insert Config"

AES - Alarm Screen : SysMgm - Diagnostics - Alarmscreen
Do not configure columns in AES Settings.

\subsection{Others configs}
\begin{itemize}
    \item WinCC OA value range config
Specifies the “valid” range for its specific dpe.
If “original value” goes out of range, it is flagged “invalid”. 
    \item Default value
Specifies an always “valid” value for its specific dpe
If “original value” goes out of range, the “online value” takes on this default value instead.
    \item User range config
Specifies the range of values that a user (holding a specific authorization) is permitted to
write into this DPE. Useful for restricting manual data entry possibilities.
    \item DP function config
Defines the original value of a DPE in terms of the values of other DPEs, e.g.(DPE1+DPE2)/2
Various statistical functions are available, e.g. take on the value of an input DP averaged over the last hour or last day. Executed in the Event Manager
    \item Smoothing - Dangerous, as smoothing include throwing data
    \item Authorization config
    \item Etc...
\end{itemize}


%\section{Alternative way to start UIs using the WinCC OA http built-in functionality}

%\section{Data Visualization with fwWebViewPLus and highcharts.js}

\newpage

\section{Logbook}
\subsection*{Monday 4th May}
\begin{itemize}
    \item Starting work at 11.00 with meeting with Luis.
    \item Reading documentation about WinCC-OA, trying to set up.
    \item Sending emails.
    \item Ending around 17.30
\end{itemize}

\subsection*{Tuesday 5th May}
\begin{itemize}
    \item Starting work at 9.30
    \item Meeting with Luis at 11.00 
    \item Finalizing setup, starting tutorials from the documentation.
    \item Helping Loann ;)
    \item Starting Exercice 1 with Loann, but troubles to make it work... :'(
    \item Ending around 17.30
\end{itemize}

\subsection*{Wednesday 6th May}
\begin{itemize}
    \item Starting work at 7.50
    \item Exercice 2 Done.
    \item Exercice 3 Done.
    \item Exercice 4 Done.
    \item Slides 1.pdf finished ! Yay ! 214 slides completed.
    \item Exceptional Pause at 12.00 to 14.00 for phone call.
    \item Started Slides 2.pdf
    \item Ending at 17.20
\end{itemize}

\subsection*{Thursday 7th May}
\begin{itemize}
    \item Starting work at 9.00
    \item Issues with home WiFi...
    \item Exercice 5-6. Some difficulties with the CTRL system.
    \item Pause at 13.00
\end{itemize}


\end{document}